\chapter{System Architecture}

\section{Protocols and node design}
There are two protocols which support different sets of mini-protocols:
\begin{itemize}
  \item\hyperref[section:node-to-node-protocol]{node-to-node protocol}
    for communication between different nodes usually run by different
    entities across the globe.  It consists of
    \hyperref[chain-sync-protocol]{chain-sync},
    \hyperref[block-fetch-protocol]{block-fetch},
    \hyperref[tx-submission-protocol]{tx-submission} and
    \hyperref[keep-alive-protocol]{keep-alive} mini-protocols.
  \item\hyperref[section:node-to-client-protocol]{node-to-client
    protocol} for intra-process communication, which allows to build
    applications that need access to the blockchain, ledger, e.g. a wallet, an
    explorer, etc.  It consists of \hyperref[chain-sync-protocol]{chain-sync},
    \hyperref[local-tx-submission-protocol]{local-tx-submission} and
    \hyperref[local-state-query-protocol]{local-state-query} mini-protocols.
\end{itemize}

Chain-sync mini-protocol (the node-to-node version) is used to replicate
a remote chain of headers; block-fetch mini-protocol to download blocks and
tx-submission to disseminate transactions across the network.

\begin{figure}
  \pgfdeclareimage[height=15cm]{node-diagram-chains-state}{figure/node-diagram-concurrency.pdf}
  \begin{center}
    \pgfuseimage{node-diagram-chains-state}
  \end{center}
  \caption{Cardano Node}
  \label{node-diagram-concurrency}
\end{figure}

Figure~\ref{node-diagram-concurrency} illustrates the design of a node.  Circles
represents threads that run one of the mini-protocols.
Each mini-protocols communicate with a remote node over the network.
Threads communicate by means of shared mutable variables, which
are represented by boxes in Figure~\ref{node-diagram-concurrency}.
We heavily use
\href{https://en.wikipedia.org/wiki/Software_transactional_memory}{Software
transactional memory} (STM), which is a mechanism for safe and lock-free
concurrent access to mutable state (see \cite{stm:harris2006}).

The ouroboros-network supports multiplexing mini-protocols, which allows us to run
the node-to-node or the node-to-client protocol on a single bearer, e.g. a TCP
connection; other bearers are also supported.  This means that chain-sync,
block-fetch and tx-submission mini-protocols will share a single TCP
connection.  The multiplexer and its framing are described in
Chapter~\ref{chapter:multiplexer}.

\section{Congestion Control}
A central design goal of the system is robust operation at high workloads.  For
example, it is a normal working condition of the networking design that
transactions arrive at a higher rate than the number that can be included in
blockchain.  An increase in the rate at which transactions are submitted must
not cause a decrease in the blockchain quality.

Point-to-point TCP bearers do not deal well with overloading.  A TCP connection
has a certain maximal bandwidth, i.e. a certain maximum load that it can handle
relatively reliably under normal conditions.  If the connection is ever
overloaded, the performance characteristics will degrade rapidly unless the
load presented to the TCP connection is appropriately managed.

At the same time, the node itself has a limit on the rate at which it can
process data.  In particular, a node may have to share its processing power
with other processes that run on the same machine/operation system instance,
which means that a node may get slowed down for some reason, and the system may
get into a situation where there is more data available from the network than the
node can process.  The design must operate appropriately in this situation and
recover from transient conditions.  In any condition, a node must not exceed
its memory limits, that is there must be defined limits, breaches of which
are treated like protocol violations.

Of course, it makes no sense if the system design is robust but so defensive
that it fails to meet performance goals.  An example would be a protocol that
never transmits a message unless it has received an explicit ACK for the
previous message. This approach might avoid overloading the network but would
waste most of the potential bandwidth.  To avoid such performance problems our
implementation relies upon protocol pipelining.


\section{Real-time Constraints and Coordinated Universal Time}
Ouroboros models the passage of physical time as an infinite sequence of time
slots, i.e. contiguous, equal-length intervals of time, and assigns slot
leaders (nodes that are eligible to create a new block) to those time slots.
At the beginning of a time slot, the slot leader selects the blockchain and
transactions that are the basis for the new block, then it creates the new
block and sends the new block to its peers.  When the new block reaches the
next block leader before the beginning of the next time slot, the next block leader
can extend the blockchain upon this block (if the block did not arrive on time
the next leader will create a new block anyway).

There are some trade-offs when choosing the slot time that is used for the
protocol, but basically, the slot length should be long enough such that a new
block has a good chance of reaching the next slot leader in time.  It is assumed
that the clock skews between the local clocks of the nodes is small with
respect to the slot length.

However, no matter how accurate the local clocks of the nodes are with respect
to the time slots, the effects of a possible clock skew must still be carefully
considered.  For example, when a node time-stamps incoming blocks with its
local clock time, it may encounter blocks that are created in the future with
respect to the local clock of the node.  The node must then decide whether this
is because of a clock skew or whether the node considers this as adversarial
behaviour of another node.

\section{Nodes behind NAT firewalls}

Many end-user systems, as well as enterprise systems, operate behind firewalls,
in particular
\href{https://en.wikipedia.org/wiki/Network_address_translation}{NATs}.  Such
environments require specific support from the software to operate
a P2P distributed system on equal terms. One of the requirements for
\texttt{cardano-node} was for nodes operating behind a NAT to be able to
contribute to the network. This includes full-node wallets (e.g. Daedalus) or
enterprise deployments of relays/BPs to supply blocks, instead of only
submitting transactions and consuming the blockchain.  This requires a way for
the outside world to traverse through the NAT. \texttt{ouroboros-network} uses
a hole-punching method for this purpose.

\subsection{Implementation details}
\texttt{ouroboros-network} allows for promoting an inbound (or outbound)
connection to be used as an outbound (inbound) connection.  This way, an
outgoing connection from behind a NAT can be reused as an inbound one, allowing
access to blocks produced (or relayed) behind the firewall.  Collected data
shows that such nodes (primarily coming from Daedalus users) meaningfully
contribute to the network.

\texttt{ouroboros-network} goes one step forward and binds all outbound
connections to the same port as inbound connections (with some exceptions,
controlled by a topology file).  This allows us to reduce the number of open
file-descriptors used by the node (at most by the number of all established
outbound connections, e.g. 70 by default, and much more for some specific
configurations).
