\chapter{Time and size limits}

\section{Timeouts}
\label{section:timeouts}

There are several layers, where timeouts play a crucial way in making the
system secure.  At the lowest layer is a mux timeout which we explain next.
After establishing a connection (either a Node-to-Node or Node-to-Client one),
the handshake is using a bearer with \texttt{10s} timeout on receiving each
mux SDU.  Note, this is a timeout which bounds how long it takes to receive
a single mux SDU, e.g. from receiving the leading edge of the mux SDU until
receiving its trailing edge, not how long we wait to receive a next SDU.
Handshake protocol is then imposing its own timeouts, see
table~\ref{table:handshake-timeouts}.

After handshake negotiation is done, mux is using a bearer with \texttt{30s}
timeout on receiving a mux SDU (the previous note applies as well). Once
a mini-protocol is in execution it must enforce it's own set of timeouts
which we included in the previous chapter and for convenience we referenced
them in the table~\ref{Node-To-Node-timeouts} below.

\begin{figure}[ht]
  \begin{center}
    \begin{tabular}{ll}
      \header{mini-protocol} & \header{timeouts} \\\hline
      Handshake              & table~\ref{table:handshake-timeouts} \\
      Chain-Sync             & table~\ref{table:chain-sync-timeouts} \\
      Block-Fetch            & table~\ref{table:block-fetch-timeouts} \\
      Tx-Submission          & table~\ref{table:tx-submission-timeouts} \\
      Keep-Alive             & table~\ref{table:keep-alive-timeouts} \\
      Peer-Share             & table~\ref{table:peer-share-timeouts}
    \end{tabular}
    \caption{Node-To-Node mini-protocol timeouts}
    \label{Node-To-Node-timeouts}
  \end{center}
\end{figure}

On the inbound side of the Node-to-Node protocol, we also include a \texttt{5s}
idleness timeout.  It starts either when a connection is accepted or when all
responder mini-protocols terminated.  If this timeout expires, without
receiving any message from a remote end, the connection must be closed unless
it is a duplex connection which is used by the outbound side.

Once all outbound and inbound mini-protocols have terminated and the idleness
timeout expired, the connection is reset and put on a \texttt{60s} timeout.
See section~\ref{sec:connection-close} why this timeout is required.
% TODO: if connection-manager is removed from this document, we need to
% preserve `sec:connection-close`.

% TODO
\section{Space limits}
All per mini-protocol size limits are referenced in table~\ref{Node-To-Node-size-limits}:
\begin{figure}[ht]
  \begin{center}
    \begin{tabular}{ll}
      \header{mini-protocol} & \header{space limits} \\\hline
      Handshake              & table~\ref{table:handshake-size-limits} \\
      Chain-Sync             & table~\ref{table:chain-sync-size-limits} \\
      Block-Fetch            & table~\ref{table:block-fetch-size-limits} \\
      Tx-Submission          & table~\ref{table:tx-submission-size-limits} \\
      Keep-Alive             & table~\ref{table:keep-alive-size-limits} \\
      Peer-Share             & table~\ref{table:peer-share-size-limits}
    \end{tabular}
    \caption{Node-To-Node mini-protocol size limits}
    \label{Node-To-Node-size-limits}
  \end{center}
\end{figure}
